\documentclass[12pt]{beamer}

\usetheme{Madrid}
\usecolortheme{seahorse}
\usefonttheme{professionalfonts}

\usepackage[T2A]{fontenc}
\usepackage[utf8]{inputenc}
\usepackage[russian]{babel}
\usepackage{amsmath,amssymb,bm}
\usepackage{booktabs}
\usepackage{tikz}
\usetikzlibrary{trees,positioning}

\title[Решающие деревья]{Решающие деревья: простая и понятная модель}
\author{Лазар В. И., Козлова Е. Р.}
\date{\today}

\begin{document}

%----------------------------- Title ---------------------------------
\begin{frame}
	\titlepage
\end{frame}

%----------------------------- Plan ----------------------------------
\section{Интуиция и постановка задач}

\begin{frame}{Зачем дерево?}
	\begin{itemize}
		\item \textbf{Идея дерева}: задаём последовательность простых вопросов вида «Признак $x_j$ больше порога?».
		\item Итог — путь от \textbf{корня} к \textbf{листу} даёт решение.
	\end{itemize}
\end{frame}

\begin{frame}{Части решающего дерева}
	\begin{itemize}
		\item \textbf{Узел} — проверка правила (например, $x_j \le t$).
		\item \textbf{Рёбра} — варианты ответа («да»/«нет»).
		\item \textbf{Лист} — предсказание: класс (классификация) или число (регрессия).
		\item \textbf{Глубина} — максимальное число вопросов от корня до листа.
	\end{itemize}
\end{frame}

\section{Наглядный пример}

\begin{frame}{Мини-пример дерева (погода и прогулка)}
	\centering
	\begin{tikzpicture}[
			level 1/.style={sibling distance=48mm, level distance=16mm},
			level 2/.style={sibling distance=32mm, level distance=18mm},
			every node/.style={align=center, font=\small},
			decision/.style={draw, rounded corners, fill=blue!10, inner sep=3pt},
			action/.style={draw, rounded corners, inner sep=3pt}
		]
		\node[decision]{Осадки?}
		child { node[action, fill=green!10]{Взять зонт\\и идти гулять}
				edge from parent node[left]{да}}
		child { node[decision]{Ветер $> 7$ м/с?}
				child { node[action, fill=red!10]{Остаться дома}
						edge from parent node[left]{да}}
				child { node[action, fill=green!10]{Идти гулять}
						edge from parent node[right]{нет}}
				edge from parent node[right]{нет}};
	\end{tikzpicture}
\end{frame}

\section{Инференс (предсказание)}

\begin{frame}{Как дерево делает предсказание (инференс)}
	\textbf{Шаги:}
	\begin{enumerate}
		\item Стартуем в корне.
		\item Проверяем условие в узле (например, $x_j \le t$).
		\item Идём по ветке «да» или «нет» в следующий узел.
		\item Повторяем, пока не попадём в лист.
		\item В листе берём предсказание: \\
		      \quad классификация — самый частый класс в листе; \\
		      \quad регрессия — среднее значение ответов в листе.
	\end{enumerate}
	\vspace{3mm}
	\textbf{Сложность:} $O(\text{глубина})$ проверок на один объект.
\end{frame}

\section{Качество разбиений}

\begin{frame}{Критерии для классификации}
	Пусть в узле есть выборка $S$ с долями классов $p_1,\dots,p_K$.
	\begin{block}{Индекс Джини}
		\[
			\mathrm{Gini}(S) = 1 - \sum_{k=1}^{K} p_k^2
		\]
	\end{block}
	\begin{block}{Энтропия}
		\[
			H(S) = - \sum_{k=1}^{K} p_k \log_2 p_k
		\]
	\end{block}
\end{frame}
\begin{frame}
	\begin{block}{Прирост информации (качество сплита по признаку $x_j$ и порогу $t$)}
		Пусть разбиение даёт дочерние множества $S_L, S_R$. Тогда
		\[
			\Delta = \mathrm{Impurity}(S) - \frac{|S_L|}{|S|}\mathrm{Impurity}(S_L) - \frac{|S_R|}{|S|}\mathrm{Impurity}(S_R),
		\]
		где $\mathrm{Impurity}$ — Gini или энтропия. Чем больше $\Delta$, тем лучше сплит.
	\end{block}
\end{frame}

\begin{frame}{Критерии для регрессии}
	Пусть ответы в узле — $y_1,\dots,y_n$, среднее $\bar y$.
	\begin{block}{Среднеквадратичная ошибка (MSE) в узле}
		\[
			\mathrm{MSE}(S) = \frac{1}{|S|}\sum_{i=1}^{|S|} (y_i - \bar y)^2
		\]
	\end{block}
	\begin{block}{Снижение ошибки при сплите}
		\[
			\Delta = \mathrm{MSE}(S) - \frac{|S_L|}{|S|}\mathrm{MSE}(S_L) - \frac{|S_R|}{|S|}\mathrm{MSE}(S_R).
		\]
		Выбираем признак и порог с максимальным $\Delta$.
	\end{block}
\end{frame}

\section{Алгоритм обучения}

\begin{frame}{Идея обучения (жадно, сверху вниз)}
	\begin{enumerate}
		\item В текущем узле перебираем \textbf{кандидатные сплиты}: признаки и пороги.
		\item Считаем \textbf{снижение нечистоты} $\Delta$ (Gini/энтропия для классификации, MSE для регрессии).
		\item Берём \textbf{лучший} сплит. Разделяем выборку на $S_L$ и $S_R$.
		\item Рекурсивно повторяем для дочерних узлов.
		\item Останавливаемся по критериям: достигнута \textbf{макс. глубина}, мало объектов в узле, \textbf{улучшение незначительно} и т.п.
		\item (Опционально) \textbf{Обрезаем} дерево (pruning), чтобы уменьшить переобучение.
	\end{enumerate}
\end{frame}

\begin{frame}
	\small
	% \begin{block}{}
	% 	\textbf{Вход:} обучающая выборка $S$, список признаков $\mathcal{F}$, параметры (\texttt{max\_depth}, \texttt{min\_samples\_leaf}, \dots)\\
	% 	\textbf{Выход:} решающее дерево $T$
	% \end{block}

	\begin{enumerate}
		\item Функция \textsc{GrowNode}$(S, depth)$:
		      \begin{enumerate}
			      \item Если \texttt{depth} $\ge$ \texttt{max\_depth} \textbf{или} $|S| <$ \texttt{min\_samples\_leaf} \textbf{или} все объекты одного класса:
			            \begin{itemize}
				            \item создать \textbf{лист} с предсказанием (мода класса / среднее $y$).
				            \item вернуть лист.
			            \end{itemize}
			      \item Иначе: для каждого признака $x_j \in \mathcal{F}$ и порога $t$ из кандидатов:
			            \begin{itemize}
				            \item разбить $S$ на $S_L=\{x: x_j \le t\}$ и $S_R=\{x: x_j > t\}$,
				            \item вычислить $\Delta(x_j,t)$ (снижение нечистоты).
			            \end{itemize}
			      \item Выбрать $(j^*, t^*) = \arg\max \Delta(x_j,t)$. Если улучшение слишком мало — сделать лист.
			      \item Рекурсивно построить детей: \textsc{GrowNode}$(S_L, depth+1)$ и \textsc{GrowNode}$(S_R, depth{+}1)$.
			      \item Вернуть внутренний узел $(x_{j^*}\le t^*)$ с привязанными детьми.
		      \end{enumerate}
		\item Корень: \textsc{GrowNode}$(S,0)$.
		\item (Опц.) Пост-обрезка по стоимости сложности: минимизируем $R_\alpha(T)=R(T)+\alpha|T|$.
	\end{enumerate}
\end{frame}

\section{Практические аспекты}

\begin{frame}{Работа с признаками}
	\begin{itemize}
		\item \textbf{Числовые:} ищем пороги $t$ (часто по отсортированным значениям).
		\item \textbf{Категориальные:} бинарные сплиты по подмножествам категорий (или one-hot кодирование).
		\item \textbf{Пропуски:} простой вариант — заполнять медианой/модой; продвинутый — surrogate splits.
		\item \textbf{Масштабирование:} деревьям в целом не требуется нормировка признаков.
	\end{itemize}
\end{frame}

\begin{frame}{Переобучение и как с ним бороться}
	\begin{itemize}
		\item Глубокие деревья могут \textbf{запоминать} данные.
		\item \textbf{Пред-обрезка} (pre-pruning): ограничить \texttt{max\_depth}, \texttt{min\_samples\_leaf}, \texttt{min\_impurity\_decrease}.
		\item \textbf{Пост-обрезка} (post-pruning): удалять ветви, если они почти не улучшают качество (например, по $R_\alpha(T)$).
		\item \textbf{Валидация}: отложенная выборка или кросс-валидация, чтобы выбрать параметры.
	\end{itemize}
\end{frame}

\begin{frame}{Сложность и масштабируемость}
	\begin{itemize}
		\item \textbf{Обучение:} при аккуратной реализации около $O(n\,d\log n)$, где $n$ — объектов, $d$ — признаков.
		\item \textbf{Инференс:} $O(\text{глубина})$ проверок на объект.
		\item Хорошо работают на табличных данных, устойчивы к монотонным преобразованиям признаков.
	\end{itemize}
\end{frame}

\begin{frame}{Плюсы и минусы}
	\begin{columns}[T]
		\begin{column}{0.48\textwidth}
			\textbf{Плюсы}
			\begin{itemize}
				\item Понятны и интерпретируемы.
				\item Не требуют масштабирования.
				\item Умеют смешивать числовые и категориальные признаки.
				\item Быстрый инференс.
			\end{itemize}
		\end{column}
		\begin{column}{0.48\textwidth}
			\textbf{Минусы}
			\begin{itemize}
				\item Склонность к переобучению без обрезки.
				\item Нестабильность при малых изменениях данных.
				\item Пороги — кусочно-постоянные решения (ступеньки).
			\end{itemize}
		\end{column}
	\end{columns}
\end{frame}

\section{Мини-пример расчёта}

\begin{frame}{Пример: классификация (индекс Джини)}
	В узле 20 объектов: 12 «положительный» и 8 «отрицательный» классы. \\
	\[
		\mathrm{Gini}(S) = 1 - \left(\frac{12}{20}\right)^2 - \left(\frac{8}{20}\right)^2
		= 1 - 0.36 - 0.16 = 0.48.
	\]
	После сплита получили:
	\[
		|S_L|=10:~(8^+,2^-) \Rightarrow \mathrm{Gini}(S_L)=1-0.64-0.04=0.32, \quad
		|S_R|=10:~(4^+,6^-) \Rightarrow \mathrm{Gini}(S_R)=1-0.16-0.36=0.48.
	\]
	Взвешенная нечистота:
	\[
		\frac{10}{20}\cdot 0.32 + \frac{10}{20}\cdot 0.48 = 0.40.
	\]
	Прирост: $\Delta = 0.48 - 0.40 = 0.08$. Чем больше $\Delta$, тем лучше сплит.
\end{frame}

\begin{frame}{Ключевые гиперпараметры}
	\begin{itemize}
		\item \texttt{criterion}: \texttt{gini} / \texttt{entropy} (классификация), \texttt{mse} (регрессия).
		\item \texttt{max\_depth}, \texttt{min\_samples\_leaf}, \texttt{min\_impurity\_decrease}.
		\item \texttt{max\_features}: ограничение числа признаков при выборе сплита.
	\end{itemize}
\end{frame}

\section{Ансамбли деревьев}

\begin{frame}{Почему ансамбли? Интуиция}
	\begin{itemize}
		\item \textbf{Один классификатор} может ошибаться по разным причинам (шум, малая выборка).
		\item \textbf{Ансамбль} объединяет несколько моделей и чаще \textbf{снижает дисперсию} (колебания) и/или \textbf{смещение}.
		\item Идея: «голосование» или «усреднение» многих деревьев даёт более устойчивый результат.
		\item Три подхода: \textbf{бэггинг} (bagging), \textbf{бустинг} (boosting), \textbf{стэкинг} (stacking).
	\end{itemize}
\end{frame}

\begin{frame}{Бэггинг (Bootstrap Aggregating)}
	\begin{itemize}
		\item Обучаем много деревьев на \textbf{бутстрэп-выборках} (случайные выборки с возвращением из обучающих данных).
		\item \textbf{Классификация}: итог — \textbf{голосование большинством}. \quad
		      \textbf{Регрессия}: итог — \textbf{среднее} предсказаний.
		\item \textbf{Эффект:} сильное снижение \textbf{дисперсии} (меньше переобучения).
		\item \textbf{OOB-оценка} (Out-of-Bag): объекты, не попавшие в бутстрэп, служат для быстрой оценки качества без отдельной валидации.
		\item \textbf{Случайный лес} = бэггинг $+$ случайный поднабор признаков при поиске сплита (\texttt{max\_features}) $\Rightarrow$ деревья становятся менее похожими, ансамбль — сильнее.
	\end{itemize}
\end{frame}

\begin{frame}{Бэггинг: ключевые параметры}
	\begin{itemize}
		\item \texttt{n\_estimators}: число деревьев (обычно десятки/сотни).
		\item \texttt{max\_depth}, \texttt{min\_samples\_leaf}: сложность базовых деревьев.
		\item \texttt{max\_features}: доля/число признаков для сплита (случайный лес).
		\item \texttt{bootstrap}: брать ли бутстрэп-выборки (обычно да).
		\item \textbf{Плюсы:} простота, устойчивость, хорошая база по умолчанию.
		\item \textbf{Минусы:} чуть меньшая интерпретируемость, скорость инференса ниже (много деревьев).
	\end{itemize}
\end{frame}

\begin{frame}{Стэкинг (Stacking): «модель над моделями»}
	\begin{itemize}
		\item Обучаем \textbf{разные} базовые модели (например, дерево, kNN, логистическую регрессию).
		\item Формируем \textbf{признаки второго уровня}: предсказания базовых моделей.
		\item Обучаем \textbf{мета-модель} на этих признаках, чтобы \textbf{научиться комбинировать} их лучше простого среднего.
		\item Чтобы избежать «подглядывания», используем \textbf{OOF-предсказания}: K-fold разбиение, на каждом фолде базовые модели предсказывают на валидации; склеиваем OOF-стек-признаки и учим мета-модель.
	\end{itemize}
\end{frame}

\begin{frame}{Схема стэкинга (OOF)}
	\centering
	\begin{tikzpicture}[
			node distance=8mm and 12mm,
			every node/.style={font=\small},
			mdl/.style={draw, rounded corners, fill=blue!10, inner sep=3pt},
			meta/.style={draw, rounded corners, fill=green!10, inner sep=3pt},
			io/.style={draw, rounded corners, inner sep=3pt},
			% >=Latex
		]
		\node[io] (X) {$\mathbf{X}$, признаки};
		\node[mdl, below left=12mm and 18mm of X] (m1) {Модель 1};
		\node[mdl, below=12mm of X] (m2) {Модель 2};
		\node[mdl, below right=12mm and 18mm of X] (m3) {Модель 3};
		\draw[->] (X) -- (m1);
		\draw[->] (X) -- (m2);
		\draw[->] (X) -- (m3);

		\node[io, below=16mm of m2] (Z) {OOF-признаки: $\hat y^{(1)},\hat y^{(2)},\hat y^{(3)}$};
		\draw[->] (m1) -- (Z);
		\draw[->] (m2) -- (Z);
		\draw[->] (m3) -- (Z);

		\node[meta, below=14mm of Z] (meta) {Мета-модель};
		\draw[->] (Z) -- (meta);

		\node[io, below=12mm of meta] (yhat) {Итоговое предсказание};
		\draw[->] (meta) -- (yhat);
	\end{tikzpicture}
\end{frame}

\begin{frame}{Когда что выбирать? (кратко)}
	\begin{itemize}
		\item \textbf{Случайный лес (бэггинг)} — сильная «база без тюнинга», устойчив к шуму, быстро стартует.
		\item \textbf{Бустинг} — лидер по качеству на табличных задачах при грамотной настройке.
		\item \textbf{Стэкинг} — когда есть \textbf{разнородные} сильные базовые модели, и вы хотите выжать максимум.
	\end{itemize}
\end{frame}

\begin{frame}{Ансамбли: плюсы/минусы и гиперпараметры}
	\begin{columns}[T]
		\begin{column}{0.48\textwidth}
			\textbf{Плюсы}
			\begin{itemize}
				\item Выше качество, чем у одного дерева.
				\item Снижение дисперсии (бэггинг) и/или смещения (бустинг).
				\item Гибкость стэкинга.
			\end{itemize}
		\end{column}
		\begin{column}{0.48\textwidth}
			\textbf{Минусы}
			\begin{itemize}
				\item Меньше интерпретируемость.
				\item Дольше обучение и инференс.
				\item Нужна аккуратная валидация (особенно для стэкинга).
			\end{itemize}
		\end{column}
	\end{columns}
	\vspace{2mm}
	\textbf{Частые параметры:} \texttt{n\_estimators}, \texttt{learning\_rate}, \texttt{max\_depth}, \texttt{max\_features}, \texttt{subsample}/\texttt{colsample}.
\end{frame}
\end{document}
