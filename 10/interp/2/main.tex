% !TEX program = pdflatex
\documentclass[aspectratio=169]{beamer}

%------------------------------------------------
% Packages
%------------------------------------------------
\usepackage[utf8]{inputenc}
\usepackage[T2A]{fontenc}
\usepackage[russian]{babel}
\usepackage{xcolor}
\usepackage{listings}
\usepackage{tikz}
\usetheme{Madrid}

%------------------------------------------------
% Listings
%------------------------------------------------
\lstset{
  language=Python,
  basicstyle=\ttfamily\scriptsize,
  keywordstyle=\color{blue},
  commentstyle=\itshape\color{gray},
  stringstyle=\color{red},
  showstringspaces=false,
  frame=single,
  breaklines=true
}

%------------------------------------------------
% Metadata
%------------------------------------------------
\title[Лекция 2]{От токенов к абстрактному синтаксическому дереву\\ \small Лекция 2 — парсер и AST в PLY}
\author{Лазар В. И.}
\date{\today}

%------------------------------------------------
\begin{document}

%------------------------------------------------
\begin{frame}
	\titlepage
\end{frame}

%------------------------------------------------
\begin{frame}{Где мы сейчас?}
	\begin{enumerate}
		\item Лекция 1 — лексер: получили упорядоченный список токенов.
		\item \textbf{Лекция 2 — строим AST и учимся его выполнять.}
		\item Дальше — расширяем язык и пишем REPL.
	\end{enumerate}
\end{frame}

%------------------------------------------------
\begin{frame}{Что такое AST?}
	\begin{block}{Определение}
		\textbf{Абстрактное синтаксическое дерево (AST)} — структурное представление программы, в котором опущены лишние скобки и пунктуация, а каждый узел описывает семантическую сущность (число, переменную, операцию…).
	\end{block}

	Почему не оставить «сырые» токены?
	\begin{itemize}
		\item Деревом проще анализировать и исполнять.
		\item Исчезают неоднозначности приоритетов.
		\item Можно легко добавлять новые фишки (<<гуляем>> по дереву).
	\end{itemize}
\end{frame}

%------------------------------------------------
\begin{frame}[fragile]{Пример 1: выражение $1 + 2 * 3$}
	\vspace{-0.8em}
	\begin{columns}[c]
		\column{0.55\textwidth}
		\begin{tikzpicture}[level distance=1cm,sibling distance=2.5cm, every node/.style={font=\small}]
			\node{\textbf{+}}
			child{ node{\textbf{1}} }
			child{ node{\textbf{*}}
					child{ node{\textbf{2}} }
					child{ node{\textbf{3}} }};
		\end{tikzpicture}
		\column{0.45\textwidth}
		В Python‑коде это будет:
		\begin{lstlisting}
BinOp("+",
      Number(1),
      BinOp("*", Number(2), Number(3)))
\end{lstlisting}
	\end{columns}
\end{frame}

%------------------------------------------------
\begin{frame}[fragile]{Пример 2: условие}
	\begin{lstlisting}[language=Python]
If(
  cond=BinOp("<", Identifier("x"), Number(0)),
  then_block=[Print(Identifier("x"))],
  else_block=[Print(Number(0))]
)
\end{lstlisting}

	\textbf{Важно:} дерево отражает логику, а не текстовую форму.
\end{frame}

%------------------------------------------------
\begin{frame}[fragile]{Как PLY помогает строить AST}
	\begin{itemize}
		\item Лексер (\lstinline|lexer.py|) уже выдаёт токены.
		\item Теперь пишем правила \texttt{ply.yacc}: «что после чего идёт».
		\item Каждое правило — функция \lstinline|p_...|, у которой \lstinline|p[0]| возвращает узел.
	\end{itemize}
	\pause
	\begin{lstlisting}
from tinypy.ast_nodes import Number, BinOp

def p_expr_binop(p):
    """expr : expr PLUS expr
            | expr MINUS expr"""
    p[0] = BinOp(p[2], p[1], p[3])
\end{lstlisting}
\end{frame}

%------------------------------------------------
\begin{frame}[fragile]{Файл \texttt{ast\_nodes.py}}
	\begin{lstlisting}
from dataclasses import dataclass
class Node: pass
@dataclass
class Number(Node):
    value: float
@dataclass
class Identifier(Node):
    name: str
@dataclass
class BinOp(Node):
    op: str
    left: Node
    right: Node
\end{lstlisting}
\end{frame}

%------------------------------------------------
\begin{frame}[fragile]{Приоритеты и ассоциативность}
	В PLY они задаются таблицей:
	\begin{lstlisting}
precedence = (
  ('left', 'PLUS', 'MINUS'),
  ('left', 'STAR', 'SLASH'),
  ('right', 'POWER'),   # **
)
\end{lstlisting}
	Это избавляет от ручного разбора вложенных выражений.
\end{frame}

%------------------------------------------------
\begin{frame}[fragile]{Отладка парсера}
	\begin{itemize}
		\item \lstinline|parser = yacc.yacc(debug=True)| создаёт файл \texttt{parsetab.py}.
		\item Команда \lstinline|parser.parse(src, lexer=lexer)| возвращает корневой узел.
		\item Можно распечатать дерево функцией \lstinline|print(ast)| — dataclass делает это красиво.
	\end{itemize}
\end{frame}

%------------------------------------------------
\begin{frame}[fragile]{Быстрое исполнение AST}
	Идея: рекурсивно обойти дерево.
	\begin{lstlisting}
def eval_node(node, env):
    if isinstance(node, Number):
        return node.value
    if isinstance(node, Identifier):
        return env[node.name]
    if isinstance(node, BinOp):
        a = eval_node(node.left, env)
        b = eval_node(node.right, env)
        return a + b if node.op == '+' else a - b
\end{lstlisting}
	`env` — обычный словарь переменных.
\end{frame}

%------------------------------------------------
\begin{frame}[fragile]{Мини‑пример «всё вместе»}
	\begin{lstlisting}[language=Python]
src = "let x = 2 * 3; x + 1"
lexer.input(src)
ast = parser.parse(lexer=lexer)  # AST
env = {}
for stmt in ast:
    print(eval_node(stmt, env))
\end{lstlisting}
\end{frame}

%------------------------------------------------
\begin{frame}[fragile]{Практика к семинару 2}
	\begin{enumerate}
		\item Завести файл \texttt{parser.py} по шаблону.
		\item Реализовать правила для: числа, идентификатора, \texttt{+ - * /}, скобок.
		\item Добавить таблицу приоритетов.
		\item Написать функцию, печатающую полученный AST.
		\item Запустить тесты \texttt{pytest tests/test\_parser\_basic.py} (появятся в репо).
	\end{enumerate}
\end{frame}

%------------------------------------------------
\end{document}
