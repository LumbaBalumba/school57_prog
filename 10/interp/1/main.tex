% !TEX program = pdflatex
\documentclass[aspectratio=169]{beamer}
%------------------------------------------------
% Packages
%------------------------------------------------
\usepackage[utf8]{inputenc}
\usepackage[T2A]{fontenc}
\usepackage[russian]{babel}
\usepackage{xcolor}
\usepackage{listings}
\usepackage{graphicx}
\usepackage{verbatim}
\usetheme{Madrid}
%------------------------------------------------
% Listings style
%------------------------------------------------
\lstset{
    language=Python,
    basicstyle=\ttfamily\footnotesize,
    keywordstyle=\color{blue}\bfseries,
    commentstyle=\color{green!50!black}\itshape,
    stringstyle=\color{red},
    numbers=left,
    numberstyle=\tiny,
    breaklines=true,
    frame=single,
    captionpos=b,
    inputencoding=utf8
}


%------------------------------------------------
% Metadata
%------------------------------------------------
\title[Лекция 1]{Устройство интерпретатора\\ \small Лекция 1 — RegEx и лексический анализ}
\author{Лазар В. И.}
\date{\today}
%------------------------------------------------
\begin{document}

%------------------------------------------------
\begin{frame}
	\titlepage
\end{frame}
%------------------------------------------------
\begin{frame}{Почему мы изучаем интерпретаторы?}
	\begin{itemize}
		\item Python, JavaScript, Lua — все интерпретируемые.
		\item Быстрый REPL‑цикл: пишем $\rightarrow$ сразу видим результат.
		\item Даже компиляторы внутри содержат интерпретирующие фазы (AST walk).
	\end{itemize}
	\pause
\end{frame}
%------------------------------------------------

\begin{frame}{Карта курса}
	\begin{enumerate}
		\item Введение, RegEx, лексический анализ
		\item Лексер на \texttt{ply.lex}
		\item Парсер на \texttt{ply.yacc} и AST
		\item Выполнение AST, окружения
		\item Условия, циклы, встроенные функции
		\item Итоговый REPL, ошибки, перспективы
	\end{enumerate}
\end{frame}
%------------------------------------------------

\begin{frame}{Мини‑проект TinyPy}
	\begin{itemize}
		\item Итог — интерпретатор языка с переменными, условиями и циклами.
		\item Оценивание по ТЗ (до 100 б.).
	\end{itemize}
\end{frame}
%------------------------------------------------

\begin{frame}{Регулярные выражения: экспресс‑повторение}
	\begin{columns}[t]
		\column{0.48\textwidth}
		\textbf{Метасимволы}
		\begin{tabular}{ll}
			\texttt{.}         & любой символ                \\
			\texttt{*}         & 0 и более                   \\
			\texttt{+}         & 1 и более                   \\
			\texttt{?}         & 0 или 1                     \\
			\texttt{[abc]}     & a \textbar{} b \textbar{} c \\
			\texttt{\^{}[abc]} & не a,b,c                    \\
			\texttt{|}         & альтернатива                \\
			\texttt{()}        & группа                      \\
		\end{tabular}
		\column{0.48\textwidth}
		\textbf{Квантификаторы}
		\begin{tabular}{ll}
			\texttt{\{n\}}   & ровно n   \\
			\texttt{\{n,\}}  & $\ge n$   \\
			\texttt{\{,m\}}  & $\le m$   \\
			\texttt{\{n,m\}} & от n до m \\
		\end{tabular}
	\end{columns}
\end{frame}
%------------------------------------------------

\begin{frame}[fragile]{От RegEx к токенам}
	\begin{itemize}
		\item Лексер сканирует поток символов и группирует их в токены.
		\item Каждый токен описывается регулярным выражением.
	\end{itemize}
	\pause
	\begin{verbatim}
        rules = [
            (r"\d+",   "NUMBER"),
            (r"[A-Za-z_][A-Za-z0-9_]*", "IDENT"),
            (r"\+", "PLUS"), (r"-", "MINUS"),
            (r"\*", "STAR"), (r"/", "SLASH"),
            (r"\(", "LPAREN"), (r"\)", "RPAREN"),
        ]
    \end{verbatim}
\end{frame}
%------------------------------------------------

\begin{frame}[fragile]{Пример токенизации строки}
	Исходная строка: \texttt{1 + 2 * (3 - 4)} \newline
	Токенизированная строка:
	\begin{verbatim}
        [NUMBER(1), PLUS, NUMBER(2), STAR, LPAREN,
        NUMBER(3), MINUS, NUMBER(4), RPAREN]
    \end{verbatim}
	\begin{itemize}
		\item Пробелы и комментарии игнорируются.
		\item Позиции токена: строка, колонка — нужны для сообщений об ошибках.
	\end{itemize}
\end{frame}
%------------------------------------------------

\begin{frame}{Почему PLY?}
	\begin{itemize}
		\item Чистый Python, проще ставить и отлаживать
		\item Поддерживает LALR(1)‑грамматики, достаточно для TinyPy.
		\item Правила задаются обычными функциями.
	\end{itemize}
\end{frame}
%------------------------------------------------

\begin{frame}{Pipeline интерпретатора}
	\centering
	% Загрузите файл pipeline.pdf в проект Overleaf или замените изображение.
	% \includegraphics[width=0.85\linewidth]{pipeline.pdf}
	\begin{small}
		Текст $\rightarrow$ \textbf{Лексер (PLY)} $\rightarrow$ токены $\rightarrow$ \textbf{Парсер (PLY)} $\rightarrow$ AST $\rightarrow$ \textbf{Выполнение}.
	\end{small}
\end{frame}
%------------------------------------------------

\begin{frame}{Практическое задание к занятию 1}
	Реализовать минимальный лексер TinyPy:
	\begin{enumerate}
		\item Склонировать репозиторий, открыть tinypy/lexer.py.
		\item Добавить правила для токенов: \texttt{NUMBER}, \texttt{IDENT}, \texttt{+ - * / ( ) \{ \}} \texttt{=} и ключевых слов \texttt{let}, \texttt{if}, \texttt{else}, \texttt{print}, \texttt{while}.
		\item Игнорировать пробелы и комментарии \texttt{\# ...}.
		\item Обновлять \texttt{lineno} и вычислять колонку.
		\item Запустить \texttt{pytest tests/test\_lexer\_basic.py} — 16 тестов должны пройти.
	\end{enumerate}
\end{frame}
%------------------------------------------------

\end{document}
