% !TEX program = pdflatex
\documentclass[12pt]{article}
\usepackage[utf8]{inputenc}
\usepackage[T2A]{fontenc}
\usepackage[russian]{babel}
\usepackage{geometry}
\geometry{margin=2.5cm}
\usepackage{enumitem, booktabs, longtable}
\usepackage{hyperref}
\hypersetup{colorlinks=true, linkcolor=blue}

\title{TinyPy — Техническое задание (упрощённая формулировка)}
\author{Школьный курс «Как работает интерпретатор»}
\date{\today}

\begin{document}
\maketitle
\tableofcontents
\newpage

% ------------------------------------------------
\section{Что такое TinyPy}
TinyPy — маленький учебный язык, похожий на сильно упрощённый Python. Он умеет:
\begin{itemize}
	\item считать числовые выражения;
	\item хранить числа в переменных;
	\item выполнять условия \texttt{if}/\texttt{else} и циклы \texttt{while};
	\item выводить данные через \texttt{print}.
\end{itemize}
Финальный результат — программа на~Python, запускаемая так:
\begin{verbatim}
python -m tinypy              # интерактивный режим
python -m tinypy prog.tny     # исполнить файл TinyPy
\end{verbatim}

% ------------------------------------------------
\section{Структура проекта}
\begin{verbatim}
tinypy/
 --- lexer.py        # превращает текст в токены
 --- parser.py       # собирает токены в дерево (AST)
 --- ast_nodes.py    # классы узлов AST
 --- interpreter.py  # выполняет дерево
 --- repl.py         # интерактивный >>-цикл
 --- errors.py       # классы ошибок (Lexer-, Syntax-, Runtime-)

tests/               # автотесты преподавателя
\end{verbatim}

% ------------------------------------------------
\section{Токены, которые обязан узнавать лексер}
\begin{longtable}{@{}llp{6cm}@{}}
	\toprule
	Категория            & Вид                                                                                                                         & Пример                          \\ \midrule
	\endhead
	Число                & цифры (можно с точкой)                                                                                                      & \texttt{12}, \texttt{3.14}      \\
	Идентификатор        & буква/\_ + буквы/цифры/\_                                                                                                   & \texttt{x}, \texttt{total\_sum} \\
	Ключевые слова       & \multicolumn{2}{l}{\texttt{let}, \texttt{if}, \texttt{else}, \texttt{while}, \texttt{true}, \texttt{false}, \texttt{print}}                                   \\
	Односимвольные знаки & \multicolumn{2}{l}{\texttt{+\; -\; *\; /\; (\; )\; \{\; \}\; ;}}                                                                                              \\
	Двухсимвольные знаки & \multicolumn{2}{l}{\texttt{==\; !=\; <\; <=\; >\; >=\; =}}                                                                                                    \\
	Комментарии          & \# до конца строки                                                                                                          & \texttt{\# это комментарий}     \\ \bottomrule
\end{longtable}
Пробелы и табы игнорируются.

% ------------------------------------------------
\section{Синтаксис TinyPy}
\subsection{Программа}
Набор команд, разделённых переводом строки или символом \texttt{;}. Пустые строки разрешены.

\subsection{Переменные}
\begin{verbatim}
let radius = 10
let area   = 3.14 * radius * radius
\end{verbatim}
\begin{itemize}
	\item Каждое объявление начинается со слова \texttt{let}.
	\item Имя можно объявить только один раз внутри одного блока.
\end{itemize}

\subsection{Выражения}
\begin{itemize}
	\item Числа, имена переменных, \texttt{true}/\texttt{false}.
	\item Операции (от старшего приоритета к младшему):
	      \[ ** \;\;\; \rightarrow \;\;\; * / \;\;\; \rightarrow \;\;\; + - \;\;\; \rightarrow \;\;\; == != < <= > >= \]
	\item Скобки \texttt{( … )} меняют порядок вычислений.
\end{itemize}

\subsection{Команда‑выражение}
Если написать выражение отдельно, результат выводится автоматически:
\begin{verbatim}
1 + 2   # печатает 3
\end{verbatim}

\subsection{Функция \texttt{print}}
\begin{verbatim}
print(1 + 2)
print(radius)
\end{verbatim}

\subsection{Условие}
\begin{verbatim}
if (x > 0) {
    print(x)
} else {
    print(0)
}
\end{verbatim}

\subsection{Цикл}
\begin{verbatim}
while (n > 0) {
    print(n)
    n = n - 1
}
\end{verbatim}

\subsection{Блоки}
Код внутри \texttt{\{ … \}} создаёт новый «вложенный» набор переменных.

% ------------------------------------------------
\section{Ошибки}
\begin{tabular}{ll}
	\toprule
	Ситуация                                & Бросить исключение      \\ \midrule
	Неизвестный символ                      & \texttt{LexerError}     \\
	Неправильная последовательность токенов & \texttt{SyntaxError\_}  \\
	Деление на 0, неизвестная переменная    & \texttt{RuntimeError\_} \\ \bottomrule
\end{tabular}

Сообщение ошибки должно содержать хотя бы номер строки.

% ------------------------------------------------
\section{Оценивание}
\begin{tabular}{lr}
	\toprule
	Раздел                            & Баллы        \\ \midrule
	Лексер (лекции 1–2)               & 25           \\
	Парсер + AST (лекция 3)           & 25           \\
	Интерпретатор (лекции 4–5)        & 40           \\
	REPL + запуск из файла (лекция 6) & 10           \\
	\textbf{Итого}                    & \textbf{100} \\ \bottomrule
\end{tabular}

Большую часть баллов выставляют автотесты; остальное — проверка кода преподавателем.

% ------------------------------------------------
\section{Инструменты}
\begin{itemize}
	\item Python 3.11, библиотека \texttt{PLY}.
	\item Тесты — \texttt{pytest}, стиль — \texttt{flake8}, проверка опечаток — \texttt{codespell}.
	\item GitHub Actions уже настроен: при каждом \texttt{push} запускаются тесты, линтер и spell‑check.
\end{itemize}

\section{Формат сдачи}
Создаётся отдельная личная ветка в \href{https://github.com/LumbaBalumba/tinypy}{репозитории}. Все изменения производятся исключительно в своей ветке. Для сдачи определённого этапа создаётся Pull Request.

\end{document}
