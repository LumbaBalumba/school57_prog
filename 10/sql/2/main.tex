\documentclass{beamer}
\usepackage[utf8]{inputenc}
\usepackage[T2A]{fontenc}
\usepackage[russian]{babel}

\usetheme{Madrid}
\usecolortheme{default}

\title{SQL: Работа с JOIN}
\subtitle{Основные виды соединений и примеры использования}
\author{Лазар В. И., Козлова Е. Р.}
\date{\today}

\begin{document}

% Титульный слайд
\begin{frame}
	\titlepage
\end{frame}

% Слайд: План занятия
\begin{frame}{План занятия}
	\tableofcontents
\end{frame}

% Секция 1: Введение в JOIN
\section{Что такое JOIN}
\begin{frame}{Общее представление о JOIN}
	\begin{itemize}
		\item \textbf{JOIN} — это операция в SQL, позволяющая объединять строки из двух (или более) таблиц
		      на основе связанного столбца между ними.
		\item Используется для того, чтобы в одном запросе получать данные сразу из нескольких таблиц.
		\item Наиболее распространены следующие виды \textbf{JOIN}:
		      \begin{itemize}
			      \item \textbf{INNER JOIN}
			      \item \textbf{LEFT (OUTER) JOIN}
			      \item \textbf{RIGHT (OUTER) JOIN}
			      \item \textbf{FULL (OUTER) JOIN}
			      \item \textbf{CROSS JOIN}
		      \end{itemize}
		\item Другие специфические варианты (например, \texttt{NATURAL JOIN}) реже применяются и не всегда поддерживаются.
	\end{itemize}
\end{frame}

% Секция 2: INNER JOIN
\section{INNER JOIN}
\begin{frame}[fragile]{INNER JOIN — соединение по совпадающим значениям}
	\begin{itemize}
		\item \textbf{INNER JOIN} возвращает только те строки, у которых совпадают значения в связанных столбцах обеих таблиц.
		\item Если хотя бы в одной таблице нет соответствующей записи,
		      то такая строка не попадёт в результирующий набор.
	\end{itemize}
	\begin{block}{Синтаксис}
		\begin{verbatim}
SELECT t1.col, t2.col
FROM table1 AS t1
INNER JOIN table2 AS t2
    ON t1.key = t2.key;
\end{verbatim}
	\end{block}
\end{frame}

\begin{frame}[fragile]{Пример использования INNER JOIN}
	\textbf{Предположим, у нас есть две таблицы:}
	\begin{itemize}
		\item \texttt{orders (order\_id, customer\_id, amount)}
		\item \texttt{customers (customer\_id, customer\_name)}
	\end{itemize}

	\textbf{Запрос:}
	\begin{verbatim}
SELECT c.customer_name, o.amount
FROM customers AS c
INNER JOIN orders AS o
    ON c.customer_id = o.customer_id
WHERE o.amount > 1000;
\end{verbatim}

	\textbf{Описание:}
	\begin{itemize}
		\item Выбирает имя покупателя и сумму заказа.
		\item Возвращает только те строки, где \texttt{customer\_id} есть и в
		      \texttt{customers}, и в \texttt{orders}, и сумма заказа > 1000.
	\end{itemize}
\end{frame}

% Секция 3: LEFT JOIN
\section{LEFT JOIN}
\begin{frame}[fragile]{LEFT (OUTER) JOIN — все строки из левой таблицы}
	\begin{itemize}
		\item \textbf{LEFT JOIN} возвращает все строки из левой таблицы (указанной после \texttt{FROM}),
		      даже если в правой таблице нет соответствующих значений.
		\item Для строк, у которых нет совпадающей записи в правой таблице, значения в столбцах правой таблицы будут \textbf{NULL}.
	\end{itemize}
	\begin{block}{Синтаксис}
		\begin{verbatim}
SELECT t1.col, t2.col
FROM table1 AS t1
LEFT JOIN table2 AS t2
    ON t1.key = t2.key;
\end{verbatim}
	\end{block}
\end{frame}

\begin{frame}[fragile]{Пример использования LEFT JOIN}
	\textbf{Те же таблицы:}
	\begin{itemize}
		\item \texttt{orders (order\_id, customer\_id, amount)}
		\item \texttt{customers (customer\_id, customer\_name)}
	\end{itemize}

	\textbf{Запрос:}
	\begin{verbatim}
SELECT c.customer_name, o.amount
FROM customers AS c
LEFT JOIN orders AS o
    ON c.customer_id = o.customer_id;
\end{verbatim}

	\textbf{Описание:}
	\begin{itemize}
		\item Возвращает все строки из \texttt{customers}, даже если в \texttt{orders} для данного
		      \texttt{customer\_id} нет записей.
		\item Если покупатель не сделал заказ, \texttt{amount} будет \textbf{NULL}.
	\end{itemize}
\end{frame}

% Секция 4: RIGHT JOIN
\section{RIGHT JOIN}
\begin{frame}[fragile]{RIGHT (OUTER) JOIN — все строки из правой таблицы}
	\begin{itemize}
		\item \textbf{RIGHT JOIN} возвращает все строки из правой таблицы,
		      даже если в левой таблице нет соответствующих значений.
		\item Аналогично \textbf{LEFT JOIN}, но фокус — на правой таблице.
		\item Не все СУБД (например, SQLite) поддерживают \texttt{RIGHT JOIN} напрямую.
	\end{itemize}
	\begin{block}{Синтаксис}
		\begin{verbatim}
SELECT t1.col, t2.col
FROM table1 AS t1
RIGHT JOIN table2 AS t2
    ON t1.key = t2.key;
\end{verbatim}
	\end{block}
\end{frame}

\begin{frame}[fragile]{Пример использования RIGHT JOIN}
	\textbf{Если \texttt{RIGHT JOIN} поддерживается:}
	\begin{verbatim}
SELECT o.order_id, c.customer_name
FROM customers AS c
RIGHT JOIN orders AS o
    ON c.customer_id = o.customer_id;
\end{verbatim}

	\textbf{Описание:}
	\begin{itemize}
		\item Возвращает все строки из \texttt{orders}.
		\item Если в \texttt{customers} нет совпадающей записи по \texttt{customer\_id},
		      то \texttt{customer\_name} будет \textbf{NULL}.
	\end{itemize}
\end{frame}

% Секция 5: FULL JOIN и CROSS JOIN
\section{FULL JOIN и CROSS JOIN}
\begin{frame}[fragile]{FULL (OUTER) JOIN — все строки из обеих таблиц}
	\begin{itemize}
		\item \textbf{FULL JOIN} возвращает все строки из обеих таблиц,
		      заполняя \textbf{NULL} там, где нет совпадающих значений.
		\item Объединяет в себе идею \textbf{LEFT JOIN} и \textbf{RIGHT JOIN}.
		\item Тоже не поддерживается во всех СУБД (в частности, в SQLite нет \textbf{FULL JOIN}).
	\end{itemize}
	\begin{block}{Синтаксис}
		\begin{verbatim}
SELECT t1.col, t2.col
FROM table1 AS t1
FULL JOIN table2 AS t2
    ON t1.key = t2.key;
\end{verbatim}
	\end{block}
\end{frame}

\begin{frame}[fragile]{CROSS JOIN — декартово произведение}
	\begin{itemize}
		\item \textbf{CROSS JOIN} возвращает декартово произведение строк:
		      каждая строка левой таблицы «скрещивается» с каждой строкой правой.
		\item При \textbf{CROSS JOIN} обычно не указывается условие \texttt{ON}, либо оно игнорируется.
	\end{itemize}
	\begin{block}{Синтаксис}
		\begin{verbatim}
SELECT t1.col, t2.col
FROM table1 AS t1
CROSS JOIN table2 AS t2;
\end{verbatim}
	\end{block}
\end{frame}

% Секция 6: Сравнение вложенных запросов и JOIN
\section{Сравнение с вложенными запросами}
\begin{frame}{Когда лучше использовать JOIN, а когда — вложенные запросы?}
	\begin{itemize}
		\item \textbf{JOIN} часто предпочтительнее по производительности при больших объёмах данных.
		\item \textbf{Вложенные запросы} удобны, когда нужно получить промежуточные вычисления
		      (например, выборку уникальных значений или агрегацию), и логичнее оформить это «внутри» основного запроса.
		\item При \textbf{JOIN} можно в одном запросе отобразить данные из нескольких таблиц
		      в виде одной «плоской» структуры.
		\item СУБД обычно эффективно оптимизируют \textbf{JOIN}, особенно если правильно настроены индексы.
	\end{itemize}
\end{frame}

\begin{frame}[fragile]{Пример: JOIN vs Subquery}
	\textbf{Пусть нужно вывести имена клиентов, сделавших заказы дороже 1000.}

	\textbf{Через вложенный запрос:}
	\begin{verbatim}
SELECT customer_name
FROM customers
WHERE customer_id IN (
    SELECT customer_id
    FROM orders
    WHERE amount > 1000
);
\end{verbatim}

	\textbf{Через JOIN:}
	\begin{verbatim}
SELECT DISTINCT c.customer_name
FROM customers AS c
JOIN orders AS o
    ON c.customer_id = o.customer_id
WHERE o.amount > 1000;
\end{verbatim}
\end{frame}

% Секция 7: Задания
\section{Задания}

% --- Задания по factbook.db ---

\begin{frame}[fragile]{Задания для factbook.db}
	\begin{itemize}
		\item Напишите запрос, возвращающий для каждой страны следующие значения: название страны, население страны, городское население страны и процентное соотношение городского населения к общему
		\item Напишите запрос, возвращающий каждой страны название, количество городов в этой стране, столицу этой страны, население столицы и процент населения столицы от общего населения
	\end{itemize}
\end{frame}

\begin{frame}[fragile]{Задания для chinook.db}
	\begin{itemize}
		\item Напишите запрос, получающие следующие колонки из объединения таблиц: invoice\_id=1, track\_id, track\_name, media\_type\_name, quantity и unit\_price
		\item Напишите запрос, получающий информацию о названии альбома, имени испонителя и общем количестве проданных копий альбома
		\item Для каждого клиента выведите суммарное количество средств на его счетах. Полученные данные отсортируйте в порядке убывания средств на счетах.
	\end{itemize}
\end{frame}

\begin{frame}[fragile]{Дополнительное задание}
	Попробуйте решить задания из предыдущего задания с помощью \texttt{JOIN} вместо вложенных запросов.
\end{frame}

\end{document}
