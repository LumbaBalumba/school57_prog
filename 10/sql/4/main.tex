\documentclass{beamer}
\usepackage[utf8]{inputenc}
\usepackage[T2A]{fontenc}
\usepackage[russian]{babel}

\usetheme{Madrid}
\usecolortheme{default}

\title{VIEW и оконные функции}
\subtitle{Создание представлений и обзор возможностей оконных функций}
\author{Лазар В. И., Козлова Е. Р.}
\date{\today}

\begin{document}

% ----------------------------------------------------
% Титульный слайд
% ----------------------------------------------------
\begin{frame}
	\titlepage
\end{frame}

% ----------------------------------------------------
% Слайд: План занятия
% ----------------------------------------------------
\begin{frame}{План занятия}
	\tableofcontents
\end{frame}

% ----------------------------------------------------
% Секция 1: VIEW
% ----------------------------------------------------
\section{Что такое VIEW и зачем он нужен}

\begin{frame}{Зачем нужны VIEW (представления)?}
	\begin{itemize}
		\item \textbf{VIEW} (представление) — виртуальная таблица, определяемая запросом \texttt{SELECT}.
		\item Хранит не данные сами по себе, а только определение запроса.
		\item Упрощает логику: можно «сохранить» сложный запрос как \texttt{VIEW} и затем работать с ним как с обычной таблицей.
		\item Повышает безопасность: даёт доступ к нужной части данных без прямого доступа к исходным таблицам.
		\item Удобно использовать для отображения объединённых или вычисляемых данных.
	\end{itemize}
\end{frame}

\begin{frame}[fragile]{Синтаксис создания VIEW}
	\begin{block}{Синтаксис (SQLite)}
		\begin{verbatim}
CREATE [OR REPLACE] VIEW view_name AS
SELECT ...
FROM ...
[WHERE ...];
\end{verbatim}
	\end{block}
	\begin{itemize}
		\item \texttt{CREATE VIEW view\_name AS <SELECT>} — создаёт новое представление.
		\item \texttt{OR REPLACE} — перезаписывает существующее представление.
		\item В SQLite представления нельзя напрямую \texttt{UPDATE}, т. к. у них нет своих данных.
	\end{itemize}
\end{frame}

\begin{frame}[fragile]{Пример: Создание VIEW}
	\textbf{Допустим, у нас есть таблицы \texttt{employees} и \texttt{departments}.}
	\begin{verbatim}
CREATE VIEW v_emp_dept AS
SELECT e.emp_id,
       e.emp_name,
       e.salary,
       d.dept_name
FROM employees AS e
JOIN departments AS d
    ON e.dept_id = d.dept_id
WHERE e.salary > 50000;
\end{verbatim}

	\textbf{Описание:}
	\begin{itemize}
		\item \texttt{v\_emp\_dept} покажет сотрудников с окладом > 50\,000 и названия их отделов.
		\item Можно теперь делать \texttt{SELECT} из \texttt{v\_emp\_dept} как из таблицы.
	\end{itemize}
\end{frame}

\begin{frame}[fragile]{Использование VIEW}
	\textbf{Как работать с созданным VIEW?}
	\begin{verbatim}
SELECT *
FROM v_emp_dept
WHERE dept_name = 'Отдел разработки';
\end{verbatim}

	\begin{itemize}
		\item Выполняется так, как будто \texttt{v\_emp\_dept} — это обычная таблица.
		\item По сути, СУБД «подставляет» исходный запрос при выполнении.
	\end{itemize}

	\textbf{Удаление VIEW:}
	\begin{verbatim}
DROP VIEW v_emp_dept;
\end{verbatim}
	\begin{itemize}
		\item Удаляет представление (не затрагивая исходные таблицы).
	\end{itemize}
\end{frame}

% ----------------------------------------------------
% Секция 2: Оконные функции
% ----------------------------------------------------
\section{Оконные функции в SQLite}

\begin{frame}{Что такое оконные функции?}
	\begin{itemize}
		\item \textbf{Оконные функции} (\textit{window functions}) — специальные функции SQL,
		      позволяющие выполнять вычисления по «окну» строк (\texttt{OVER}).
		\item В отличие от обычных агрегатных функций, не свёртывают строки в одну, а «распределяют» вычисленное значение по каждой строке.
		\item Примеры: \texttt{ROW\_NUMBER()}, \texttt{RANK()}, \texttt{LAG()}, \texttt{LEAD()}, агрегаты с \texttt{OVER}.
		\item В SQLite доступно, начиная с версии 3.25.0 (сентябрь 2018).
	\end{itemize}
\end{frame}

\begin{frame}[fragile]{Общий синтаксис оконных функций}
	\begin{verbatim}
<function>() OVER (
    [PARTITION BY column_list]
    [ORDER BY column_list]
)
\end{verbatim}

	\textbf{Пример:}
	\begin{verbatim}
SELECT
    employee_id,
    salary,
    AVG(salary) OVER (
        PARTITION BY dept_id
    ) AS avg_salary_in_dept
FROM employees;
\end{verbatim}

	\begin{itemize}
		\item \textbf{PARTITION BY} — группирует строки в «окна» (например, по отделам).
		\item \textbf{ORDER BY} — порядок, важный для функций ранжирования
	\end{itemize}
\end{frame}

\begin{frame}[fragile]{Пример 1: \texttt{ROW\_NUMBER()}}
	\begin{verbatim}
SELECT
    emp_name,
    salary,
    ROW_NUMBER() OVER (ORDER BY salary DESC) AS row_num
FROM employees;
\end{verbatim}
	\begin{itemize}
		\item \texttt{ROW\_NUMBER} даёт порядковый номер каждой строки в рамках указанного окна.
		\item В данном случае окно — вся таблица, упорядоченная по убыванию зарплаты.
	\end{itemize}
\end{frame}

\begin{frame}[fragile]{Пример 2: \texttt{RANK()} и \texttt{DENSE\_RANK()}}
	\begin{verbatim}
SELECT
    emp_name,
    salary,
    RANK() OVER (ORDER BY salary DESC) AS salary_rank,
    DENSE_RANK() OVER (ORDER BY salary DESC) AS dense_rank
FROM employees;
\end{verbatim}

	\textbf{Описание:}
	\begin{itemize}
		\item \texttt{RANK()} — при совпадении зарплат пропускает следующий ранг (1,1,3,...).
		\item \texttt{DENSE\_RANK()} — идёт подряд без пропусков (1,1,2,...).
	\end{itemize}
\end{frame}

\begin{frame}[fragile]{Пример 3: Агрегат с \texttt{PARTITION BY}}
	\textbf{Задача: для каждого сотрудника вывести его зарплату и среднюю зарплату в отделе.}
	\begin{verbatim}
SELECT
    e.emp_id,
    e.emp_name,
    e.dept_id,
    e.salary,
    AVG(e.salary) OVER (
        PARTITION BY e.dept_id
    ) AS avg_salary_in_dept
FROM employees e;
\end{verbatim}

	\begin{itemize}
		\item Каждая строка сохраняется, но получает дополнительную инфу о средней зарплате по своему отделу.
	\end{itemize}
\end{frame}

% ----------------------------------------------------
% Секция 3: Задания
% ----------------------------------------------------
\section{Задания}

\begin{frame}{Задания (1--4)}
	\begin{enumerate}
		\item \textbf{Создание VIEW для объединения таблиц:}
		      Создайте \texttt{VIEW} (например, \texttt{v\_employee\_info}) на основе \texttt{employees} и \texttt{departments},
		      выводя \texttt{emp\_name}, \texttt{salary}, \texttt{dept\_name}. Отберите только сотрудников, у которых зарплата выше среднего в их отделе
		      (подумайте, можно ли использовать оконную функцию прямо в \texttt{CREATE VIEW}).

		\item \textbf{Добавление вычисляемого поля в VIEW:}
		      Создайте представление, где помимо основных полей будет столбец «годовая зарплата» = \texttt{salary * 12}.
		      Покажите пример \texttt{SELECT} из этого представления.

		\item \textbf{VIEW с условием:}
		      Создайте представление \texttt{v\_high\_salary}, отображающее только тех сотрудников, у кого \texttt{salary} > 100000.
		      Затем сделайте \texttt{SELECT} из \texttt{v\_high\_salary} и убедитесь, что вывелись только «высокие» зарплаты.

		\item \textbf{Удаление VIEW:}
		      Удалите одно из созданных вами представлений. Убедитесь, что исходные таблицы при этом не пострадали.
	\end{enumerate}
\end{frame}

\begin{frame}{Задания (5--8)}
	\begin{enumerate}
		\setcounter{enumi}{4}
		\item \textbf{Применение оконной функции \texttt{ROW\_NUMBER()}:}
		      Напишите запрос, который присваивает каждой записи о сотруднике уникальный порядковый номер (по убыванию зарплаты) с помощью \texttt{ROW\_NUMBER()}.

		\item \textbf{\texttt{RANK()} и одинаковая зарплата:}
		      Покажите, как \texttt{RANK()} присваивает места сотрудникам, если у нескольких человек зарплата совпадает.
		      Для наглядности используйте \texttt{SELECT emp\_name, salary, RANK() OVER (ORDER BY salary DESC)}.

		\item \textbf{Агрегация с \texttt{PARTITION BY}}:
		      Для таблицы \texttt{employees} найдите, какому отделу принадлежит каждый сотрудник и какова средняя зарплата в этом отделе.
		      (Используйте \texttt{AVG(salary) OVER (PARTITION BY dept\_id)}.)

		\item \textbf{\texttt{DENSE\_RANK()} по отделам:}
		      Сгруппируйте сотрудников по отделам (\texttt{PARTITION BY dept\_id}) и упорядочьте внутри отдела по зарплате по убыванию.
		      Используйте \texttt{DENSE\_RANK()} — так каждый сотрудник увидит свой «ранг зарплаты» в своём отделе.
	\end{enumerate}
\end{frame}

\begin{frame}{Задания (9--11)}
	\begin{enumerate}
		\setcounter{enumi}{8}
		\item \textbf{Создание VIEW с оконной функцией:}
		      Создайте \texttt{VIEW}, в котором каждая строка содержит \texttt{emp\_id}, \texttt{emp\_name},
		      и позицию сотрудника (\texttt{ROW\_NUMBER()}) в общем рейтинге по зарплате.
		      Убедитесь, что при \texttt{SELECT * FROM <view\_name>} таблица корректно отображает ранги.

		\item \textbf{Сравнение зарплаты с предыдущей строкой (\texttt{LAG}):}
		      Напишите запрос, который для каждого сотрудника показывает его \texttt{salary} и «предыдущую» зарплату (\texttt{LAG(salary) OVER (ORDER BY salary)}) — так вы сможете сравнить.

		\item \textbf{Дополнительный вызов: обновление представления?}
		      Попробуйте выполнить \texttt{UPDATE} или \texttt{INSERT} через созданное \texttt{VIEW} и проверьте, даёт ли это результат или ошибку в SQLite.
	\end{enumerate}
\end{frame}

\begin{frame}{Задание 12}
	\begin{enumerate}
		\setcounter{enumi}{11}
		\item \textbf{Сложный запрос + VIEW:}
		      Попробуйте объединить данные из нескольких таблиц (например, \texttt{employees}, \texttt{departments}, \texttt{projects} — если есть)
		      в один запрос с оконной функцией, а затем сделайте из него \texttt{VIEW}, чтобы упростить повторный доступ к этим вычислениям.
	\end{enumerate}
\end{frame}

\end{document}
