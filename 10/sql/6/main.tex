\documentclass{beamer}
\usepackage[utf8]{inputenc}
\usepackage[T2A]{fontenc}
\usepackage[russian]{babel}

\usetheme{Madrid}
\usecolortheme{default}

\title{Введение в SQLAlchemy}
\subtitle{Основы ORM в Python}
\author{Лазар В. И., Козлова Е. Р.}
\date{\today}

\begin{document}

% ----------------------------------------------------
% Титульный слайд
% ----------------------------------------------------
\begin{frame}
	\titlepage
\end{frame}

% ----------------------------------------------------
% Слайд: План занятия
% ----------------------------------------------------
\begin{frame}{План занятия}
	\tableofcontents
\end{frame}

% ----------------------------------------------------
% Секция 1: Введение
% ----------------------------------------------------
\section{Что такое SQLAlchemy?}

\begin{frame}{Обзор SQLAlchemy}
	\begin{itemize}
		\item \textbf{SQLAlchemy} — одна из самых популярных ORM-библиотек для Python.
		\item Позволяет работать с разными СУБД (SQLite, PostgreSQL, MySQL и др.) единым способом.
		\item Делит функционал на \textit{Core} (работа с SQL на более низком уровне) и \textit{ORM} (объектно-реляционное отображение).
		\item Упрощает создание и сопровождение проектов: меньше «ручного» SQL, больше объектного подхода.
	\end{itemize}
\end{frame}

\begin{frame}{Зачем нужна ORM?}
	\begin{itemize}
		\item \textbf{Отделение бизнес-логики} от конкретных SQL-запросов.
		\item \textbf{Кросс-база}: без изменения кода можно подключить другую СУБД.
		\item \textbf{Удобство}: вместо таблиц и строк — привычные классы и объекты.
		\item \textbf{Безопасность}: автоматическая экранизация параметров, меньше риска SQL-инъекций.
	\end{itemize}
\end{frame}

% ----------------------------------------------------
% Секция 2: Установка и настройка
% ----------------------------------------------------
\section{Установка и настройка}

\begin{frame}[fragile]{Установка}
	\begin{block}{Используйте pip}
		\begin{verbatim}
pip install sqlalchemy
\end{verbatim}
	\end{block}

	\begin{itemize}
		\item Для работы c SQLite хватит \texttt{sqlite3} из стандартной библиотеки.
		\item При необходимости подключать драйверы для других СУБД (например, \texttt{psycopg2} для PostgreSQL).
	\end{itemize}
\end{frame}

\begin{frame}[fragile]{Базовая конфигурация}
	\begin{block}{Пример создания \texttt{engine} и \texttt{session}}
		\begin{verbatim}
from sqlalchemy import create_engine
from sqlalchemy.orm import sessionmaker

# Пример для SQLite
engine = create_engine('sqlite:///example.db')

Session = sessionmaker(bind=engine)
session = Session()
\end{verbatim}
	\end{block}

	\begin{itemize}
		\item \textbf{Engine} — «двигатель» для подключения к базе.
		\item \textbf{Session} — объект для взаимодействия с БД (объект транзакции).
	\end{itemize}
\end{frame}

% ----------------------------------------------------
% Секция 3: Создание моделей (ORM)
% ----------------------------------------------------
\section{Создание моделей}

\begin{frame}[fragile]{Декларативная модель}
	\begin{block}{Пример Python-класса}
		\begin{verbatim}
from sqlalchemy.ext.declarative import declarative_base
from sqlalchemy import Column, Integer, String

Base = declarative_base()

class User(Base):
    __tablename__ = 'users'

    id = Column(Integer, primary_key=True)
    username = Column(String, nullable=False)
    city = Column(String)
\end{verbatim}
	\end{block}

	\begin{itemize}
		\item \texttt{Base} — базовый класс для всех моделей.
		\item \textbf{Колонки} описываются как атрибуты класса.
		\item \texttt{\_\_tablename\_\_} — имя таблицы в БД.
	\end{itemize}
\end{frame}

\begin{frame}[fragile]{Создание таблиц по моделям}
	\begin{block}{Пример}
		\begin{verbatim}
Base.metadata.create_all(engine)
\end{verbatim}
	\end{block}

	\begin{itemize}
		\item \texttt{create\_all} проверяет, какие таблицы уже существуют, и создаёт отсутствующие.
		\item Полезно при первом запуске или при добавлении новых моделей.
	\end{itemize}
\end{frame}

% ----------------------------------------------------
% Секция 4: Добавление и выборка данных
% ----------------------------------------------------
\section{Работа с данными}

\begin{frame}[fragile]{Добавление данных}
	\begin{block}{Пример Python-кода}
		\begin{verbatim}
new_user = User(username='Ivan', city='Moscow')
session.add(new_user)
session.commit()
\end{verbatim}
	\end{block}

	\begin{itemize}
		\item \texttt{session.add} — подготавливает объект к вставке.
		\item \texttt{commit()} — фиксирует транзакцию (CREATE или UPDATE).
	\end{itemize}
\end{frame}

\begin{frame}[fragile]{Выборка (\texttt{SELECT})}
	\begin{block}{Пример}
		\begin{verbatim}
users = session.query(User).all()
for user in users:
    print(user.username, user.city)
\end{verbatim}
	\end{block}

	\begin{itemize}
		\item \texttt{query(User).all()} вернёт список всех объектов из таблицы \texttt{users}.
		\item Можно применять \texttt{filter}, \texttt{order\_by}, \texttt{limit} и прочие SQL-конструкции.
	\end{itemize}
\end{frame}

\begin{frame}[fragile]{Обновление данных}
	\begin{block}{Пример}
		\begin{verbatim}
user = session.query(User).filter_by(username='Ivan').first()
user.city = 'Saint Petersburg'
session.commit()
\end{verbatim}
	\end{block}

	\begin{itemize}
		\item При изменении атрибута объекта и вызове \texttt{commit()} SQLAlchemy сформирует и выполнит \texttt{UPDATE}-запрос.
	\end{itemize}
\end{frame}

\begin{frame}[fragile]{Удаление данных}
	\begin{block}{Пример}
		\begin{verbatim}
user_to_delete = session.query(User).filter_by(username='Ivan').first()
user_to_delete.delete()
session.commit()
\end{verbatim}
	\end{block}

	\begin{itemize}
		\item \texttt{session.delete} готовит объект к удалению.
		\item \texttt{commit()} выполняет \texttt{DELETE}-запрос.
	\end{itemize}
\end{frame}

% ----------------------------------------------------
% Секция 5: Связи (Relationships)
% ----------------------------------------------------
\section{Связи между моделями}

\begin{frame}[fragile]{Отношение \texttt{One-to-Many}}
	\begin{verbatim}
from sqlalchemy import ForeignKey
from sqlalchemy.orm import relationship

class Department(Base):
    __tablename__ = 'departments'
    id = Column(Integer, primary_key=True)
    name = Column(String)

class Employee(Base):
    __tablename__ = 'employees'
    id = Column(Integer, primary_key=True)
    name = Column(String)
    department_id = Column(
                Integer, ForeignKey('departments.id'))
    department = relationship(
                'Department', backref='employees')
\end{verbatim}
\end{frame}

\begin{frame}[fragile]{Отношение \texttt{One-to-Many}}
	\begin{itemize}
		\item \texttt{ForeignKey} в \texttt{Employee} указывает на \texttt{departments.id}.
		\item \texttt{relationship} позволяет обращаться к \texttt{Employee.department}.
		\item \texttt{backref='employees'} даёт доступ к списку сотрудников из \texttt{Department}.
	\end{itemize}
\end{frame}

\begin{frame}[fragile]{Отношение One-to-One}
	\begin{itemize}
		\item В SQLAlchemy нет специального ключевого слова «One-to-One».
		\item Реализуется как \texttt{One-to-Many} с дополнительным уникальным ограничением или через \texttt{uselist=False}.
	\end{itemize}

	\begin{block}{Пример c \texttt{uselist=False}}
		\begin{verbatim}
class Person(Base):
    __tablename__ = 'persons'
    id = Column(Integer, primary_key=True)
    name = Column(String)

class Passport(Base):
    __tablename__ = 'passports'
    id = Column(Integer, primary_key=True)
    person_id = Column(Integer,
        ForeignKey('persons.id'), unique=True)
    person = relationship('Person', backref='passport',
                                    uselist=False)
\end{verbatim}
	\end{block}
\end{frame}

\begin{frame}[fragile]{Отношение One-to-One}
	\begin{itemize}
		\item \texttt{unique=True} указывает, что \texttt{person\_id} уникален —
		      не может указывать на несколько записей (строк).
		\item \texttt{uselist=False} даёт понять, что ожидается единственный объект.
	\end{itemize}
\end{frame}

% ------------------- MANY TO MANY -------------------
\begin{frame}[fragile]{Отношение Many-to-Many (через ассоциативную таблицу)}
	\begin{block}{Пример}
		\begin{verbatim}
association_table = Table('association', Base.metadata,
    Column('student_id', Integer, ForeignKey('students.id')),
    Column('course_id', Integer, ForeignKey('courses.id'))
)

\end{verbatim}
	\end{block}
\end{frame}

\begin{frame}[fragile]{Отношение Many-to-Many (через ассоциативную таблицу)}
	\begin{block}{Пример}
		\begin{verbatim}
class Student(Base):
    __tablename__ = 'students'
    id = Column(Integer, primary_key=True)
    name = Column(String)
    courses = relationship('Course',
                secondary=association_table,
                back_populates='students')
class Course(Base):
    __tablename__ = 'courses'
    id = Column(Integer, primary_key=True)
    title = Column(String)
    students = relationship('Student',
                secondary=association_table,
                back_populates='courses')
\end{verbatim}
	\end{block}
\end{frame}

\begin{frame}[fragile]{Отношение Many-to-Many (через ассоциативную таблицу)}
	\begin{itemize}
		\item \texttt{association\_table} — вспомогательная таблица с полями \texttt{student\_id} и \texttt{course\_id}.
		\item \texttt{relationship(..., secondary=...)} связывает через таблицу.
		\item \texttt{back\_populates} указывает на связное поле в другой модели.
	\end{itemize}
\end{frame}
% ----------------------------------------------------
% Секция 6: Задания
% ----------------------------------------------------
\begin{frame}{Общая идея проекта}
	\textbf{Система управления проектами и задачами (консольное приложение)}:
	\begin{itemize}
		\item Написать на Python (используя SQLAlchemy) мини-систему для работы с
		      \texttt{Project}, \texttt{User}, \texttt{Profile}, \texttt{Task}.
		\item Обязательно задействовать:
		      \begin{itemize}
			      \item Все основные операции с БД (CRUD: Create, Read, Update, Delete).
			      \item Различные типы связей: One-to-One, One-to-Many, Many-to-Many.
		      \end{itemize}
	\end{itemize}
\end{frame}

\begin{frame}{Требования к проекту (1/2)}
	\begin{itemize}
		\item \textbf{Модели (таблицы) и связи:}
		      \begin{enumerate}
			      \item \textbf{User (пользователь)}
			            \begin{itemize}
				            \item \texttt{id}, \texttt{username}, \texttt{email}
				            \item Связь One-to-One с \texttt{Profile}
				            \item Связь Many-to-Many с \texttt{Project}
			            \end{itemize}
			      \item \textbf{Profile (профиль пользователя)}
			            \begin{itemize}
				            \item \texttt{id}, \texttt{bio}, \texttt{phone}, \texttt{user\_id} (уникальный ForeignKey)
			            \end{itemize}
			      \item \textbf{Project (проект)}
			            \begin{itemize}
				            \item \texttt{id}, \texttt{title}, \texttt{description}
				            \item Связь One-to-Many с \texttt{Task}
				            \item Связь Many-to-Many с \texttt{User}
			            \end{itemize}
			      \item \textbf{Task (задача)}
			            \begin{itemize}
				            \item \texttt{id}, \texttt{title}, \texttt{status}, \texttt{project\_id} (ForeignKey)
			            \end{itemize}
		      \end{enumerate}
	\end{itemize}
\end{frame}

\begin{frame}{Требования к проекту (2/2)}
	\begin{itemize}
		\item \textbf{CRUD-операции:}
		      \begin{itemize}
			      \item Создание новых записей (\texttt{User}, \texttt{Project}, \texttt{Task}, \texttt{Profile}).
			      \item Чтение (просмотр) списков и деталей (вывести всех пользователей, задачи проекта и т. п.).
			      \item Обновление (изменить \texttt{email} пользователя, статус задачи, описание проекта и т. д.).
			      \item Удаление (убрать ненужного пользователя, проект или задачу).
		      \end{itemize}
		\item \textbf{Пример функционала}: «Добавить проект», «Добавить задачу», «Вывести пользователей»,
		      «Назначить пользователя на проект» (Many-to-Many), «Добавить профиль к пользователю» (One-to-One).
		\item \textbf{Демонстрация работы связей}:
		      \begin{itemize}
			      \item \texttt{user.profile}, \texttt{project.tasks}, \texttt{user.projects}
			            и т. п. (в зависимости от настроенного \texttt{relationship}).
		      \end{itemize}
	\end{itemize}
\end{frame}

\end{document}
