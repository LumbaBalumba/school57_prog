\documentclass{beamer}
\usepackage[utf8]{inputenc}
\usepackage[T2A]{fontenc}
\usepackage[russian]{babel}

\usetheme{Madrid}
\usecolortheme{default}

\title{Использование SQLite в Python}
\subtitle{Подключение и основные операции}
\author{Лазар В. И., Козлова Е. Р.}
\date{\today}

\begin{document}

% ----------------------------------------------------
% Титульный слайд
% ----------------------------------------------------
\begin{frame}
	\titlepage
\end{frame}

% ----------------------------------------------------
% Слайд: План занятия
% ----------------------------------------------------
\begin{frame}{План занятия}
	\tableofcontents
\end{frame}

% ----------------------------------------------------
% Секция 2: Подготовка окружения
% ----------------------------------------------------
\section{Подготовка окружения}

\begin{frame}{Установка и импорт}
	\begin{itemize}
		\item \textbf{Встроенный модуль \texttt{sqlite3}} (начиная с Python 2.5).
		\item Никаких дополнительных библиотек не требуется.
		\item Импорт:
		      \begin{block}{Пример}
			      \texttt{import sqlite3}
		      \end{block}
		\item Для проверки версии SQLite используйте:
		      \begin{block}{Пример}
			      \texttt{sqlite3.sqlite\_version}
		      \end{block}
	\end{itemize}
\end{frame}

% ----------------------------------------------------
% Секция 3: Подключение и базовые операции
% ----------------------------------------------------
\section{Подключение и базовые операции}

\begin{frame}[fragile]{Создание и подключение к базе данных}
	\begin{block}{Пример кода на Python}
		\begin{verbatim}
import sqlite3

# Создаём или открываем существующий файл базы данных
conn = sqlite3.connect('example.db')

# Создаём курсор для выполнения SQL-запросов
cursor = conn.cursor()

# В конце работы не забудьте закрыть соединение
conn.close()
\end{verbatim}
	\end{block}

	\begin{itemize}
		\item Если файла \texttt{example.db} ещё нет, он будет создан автоматически.
		\item \texttt{conn} — объект соединения, \texttt{cursor} — объект для работы с SQL.
	\end{itemize}
\end{frame}

\begin{frame}[fragile]{Создание таблицы}
	\begin{block}{Пример SQL в Python}
		\begin{verbatim}
import sqlite3
conn = sqlite3.connect('example.db')
cursor = conn.cursor()
cursor.execute(''' # Создаём таблицу
    CREATE TABLE IF NOT EXISTS users (
        id INTEGER PRIMARY KEY AUTOINCREMENT,
        username TEXT NOT NULL,
        age INTEGER,
        city TEXT
    )
''')
conn.commit()  # фиксируем изменения
conn.close()
\end{verbatim}
	\end{block}

	\begin{itemize}
		\item \texttt{commit()} — фиксирует (сохраняет) изменения в базе данных.
	\end{itemize}
\end{frame}

\begin{frame}[fragile]{Добавление данных}
	\begin{block}{Пример SQL в Python}
		\begin{verbatim}
import sqlite3
conn = sqlite3.connect('example.db')
cursor = conn.cursor()
# Пример параметризированного запроса
cursor.execute('''
    INSERT INTO users (username, age, city)
    VALUES (?, ?, ?)
''', ('Иван', 25, 'Москва'))
conn.commit()
conn.close()
\end{verbatim}
	\end{block}

	\begin{itemize}
		\item Используйте \textbf{плейсхолдеры ( \texttt{?} )} и кортеж значений для защиты от SQL-инъекций.
		\item Можно выполнять \texttt{execute} несколько раз для разных данных или воспользоваться \texttt{executemany}.
	\end{itemize}
\end{frame}

\begin{frame}[fragile]{Выборка данных}
	\begin{block}{Пример SQL в Python}
		\begin{verbatim}
import sqlite3
conn = sqlite3.connect('example.db')
cursor = conn.cursor()
# Выполняем SELECT-запрос
cursor.execute('SELECT id, username, age, city FROM users')
# Получаем все результаты
rows = cursor.fetchall()
for row in rows:
    print(row)
conn.close()
\end{verbatim}
	\end{block}

	\begin{itemize}
		\item \texttt{fetchall()} — возвращает список кортежей с результатами.
		\item \texttt{fetchone()} — возвращает одну строку, \texttt{fetchmany(n)} — \texttt{n} строк.
	\end{itemize}
\end{frame}

\begin{frame}[fragile]{Обновление и удаление данных}
	\begin{block}{Обновление}
		\begin{verbatim}
cursor.execute('''
    UPDATE users
    SET age = ?
    WHERE username = ?
''', (26, 'Иван'))
\end{verbatim}
	\end{block}

	\begin{block}{Удаление}
		\begin{verbatim}
cursor.execute('''
    DELETE FROM users
    WHERE username = ?
''', ('Иван',))
\end{verbatim}
	\end{block}

	\begin{itemize}
		\item Не забудьте \texttt{commit()} после \texttt{UPDATE} или \texttt{DELETE}.
	\end{itemize}
\end{frame}

% ----------------------------------------------------
% Секция 4: Практические советы
% ----------------------------------------------------
\section{Практические советы}

\begin{frame}[fragile]{Работа с контекстными менеджерами}
	\begin{block}{Пример}
		\begin{verbatim}
import sqlite3

with sqlite3.connect('example.db') as conn:
    cursor = conn.cursor()
    cursor.execute('SELECT * FROM users')
    rows = cursor.fetchall()
    print(rows)
\end{verbatim}
	\end{block}

	\begin{itemize}
		\item С \texttt{with} не нужно явно вызывать \texttt{close()}.
		\item По выходу из блока \texttt{with} изменения автоматически \texttt{commit()}-ятся,
		      если не произошло ошибки, иначе — \texttt{rollback()}.
	\end{itemize}
\end{frame}

\begin{frame}{Организация кода в проекте}
	\begin{itemize}
		\item Вся логика работы с БД выносится в отдельный пакет (обычно его называют \texttt{db} или \texttt{models})
		\item Для каждой сущности создаётся класс, который называется моделью этой сущности. Все операции над классом преобразуются в соответствующий SQL-код
		\item Обычно в каждом методе создаётся отдельный курсор для упрощения взаимодействия, но бывают случаи, когда курсор создаётся извне и передаётся в методы (dependency injection)
	\end{itemize}
\end{frame}

% ----------------------------------------------------
% Секция 5: Итоговое задание
% ----------------------------------------------------
\section{Проектное задание}

\begin{frame}{Задание: Мини-приложение для хранения данных}
	Вы создаёте приложение для библиотеки.
	\newline
	\textbf{Требования:}
	\begin{enumerate}
		\item Хранение информации о различных книгах в библиотеке (название, автор, год выпуска, издание, номер шкафа, номер полки).
		\item Хранение информации о посетителях библиотеки (имя, фамилия, отчество, номер читательского билета, адрес).
		\item Хранение информации о том, какой пользователь какие книги читает в данный момент.
		\item Возможность внесения информации об актуальных изменениях (появление новой книги в библиотеке, регистрация нового пользователя, выдача книги пользователю, получения книги от пользователя, поиск книги по различным параметрам, и т. д.)
		\item Консольный (или графический) интерфейс для взаимодействия с программой.
	\end{enumerate}
\end{frame}

\end{document}
