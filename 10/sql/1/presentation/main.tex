\documentclass{beamer}
\usepackage[utf8]{inputenc}
\usepackage[T2A]{fontenc}
\usepackage[russian]{babel}

\usetheme{Madrid}
\usecolortheme{default}

\title{Введение в базы данных}
\subtitle{Основные определения, SQL и пример на SQLite}
\author{Лазар В. И., Козлова Е. Р.}
\date{\today}

\begin{document}

\begin{frame}
	\titlepage
\end{frame}

% Слайд 1: План занятия
\begin{frame}{План занятия}
	\tableofcontents
\end{frame}

% Секция 1: Введение
\section{Основные определения}
\begin{frame}{Что такое база данных?}
	\begin{itemize}
		\item \textbf{База данных (БД)} --- это организованная совокупность данных,
		      структурированная таким образом, чтобы упростить хранение, обработку и поиск информации.
		\item \textbf{Таблица} --- одна из основных структур хранения данных.
		      Состоит из строк (записей) и столбцов (полей).
		\item \textbf{Первичный ключ (Primary Key)} --- уникальный идентификатор для каждой строки в таблице.
		\item \textbf{СУБД (DBMS)} --- система управления базами данных (например, MySQL, PostgreSQL, SQLite).
	\end{itemize}
\end{frame}

\begin{frame}{Основные свойства баз данных}
	\begin{itemize}
		\item \textbf{Целостность данных}
		\item \textbf{Безопасность и надёжность}
		\item \textbf{Высокая производительность}
		\item \textbf{Масштабируемость}
	\end{itemize}
\end{frame}

% Секция 2: SQL и SQLite
\section{SQL и SQLite}
\begin{frame}{Что такое SQL?}
	\begin{itemize}
		\item \textbf{SQL (Structured Query Language)} --- язык структурированных запросов,
		      позволяющий взаимодействовать с базой данных.
		\item Используется для:
		      \begin{itemize}
			      \item создания и модификации схемы БД (\texttt{CREATE}, \texttt{ALTER}, \texttt{DROP});
			      \item управления данными (\texttt{INSERT}, \texttt{UPDATE}, \texttt{DELETE});
			      \item выборки данных (\texttt{SELECT}).
		      \end{itemize}
		\item \textbf{SQLite} --- лёгкая встраиваемая СУБД, которая хранит всю базу в одном файле.
	\end{itemize}
\end{frame}

\begin{frame}[fragile]{Типы данных в SQL}
	\begin{itemize}
		\item \texttt{CHAR(size)} - строка фиксированного размера size
		\item \texttt{VARCHAR(size)} - строка переменной длины длины не более size
		\item \texttt{BOOL/BOOLEAN}
		\item \texttt{SMALLINT, MEDIUMINT, INT/INTEGER} - целые числа (16, 32 и 64 бита сответственно)
		\item \texttt{FLOAT, DOUBLE} - числа с плавающей точкой размером 32 и 64 бита соответственно
		\item \texttt{DATE, TIME, DATETIME, TIMESTAMP}
	\end{itemize}
	Также в любой* колонке могут встречаться \textbf{NULL}-значения.
\end{frame}

% Секция 3: Конструкция SELECT
\section{Конструкция SELECT}
\begin{frame}{Общая форма SELECT}
	\textbf{Основной синтаксис запроса:}
	\begin{block}{Синтаксис}
		\texttt{SELECT <столбцы или выражения>\newline
			FROM <название\_таблицы>\newline
			[WHERE <условие>]\newline
			[GROUP BY <столбцы>]\newline
			[HAVING <условие\_для\_групп>]\newline
			[ORDER BY <столбцы> [ASC|DESC]]\newline
			[LIMIT <количество\_строк>];}
	\end{block}
\end{frame}

\begin{frame}
	\begin{itemize}
		\item \textbf{SELECT} указывает, какие столбцы (или вычисления) нужно вывести.
		\item \textbf{FROM} указывает, из какой таблицы получаем данные.
		\item \textbf{WHERE} --- условие фильтрации строк.
		\item \textbf{GROUP BY} --- группировка по одному или нескольким столбцам.
		\item \textbf{HAVING} --- условие для отфильтрованных групп.
		\item \textbf{ORDER BY} --- сортировка (по возрастанию или убыванию).
		\item \textbf{LIMIT} --- ограничение количества возвращаемых строк.
	\end{itemize}
\end{frame}

\begin{frame}[fragile]{Пример 1: Простой запрос}
	\textbf{Структура таблицы \texttt{employees}:}
	\begin{itemize}
		\item \texttt{id} (PRIMARY KEY)
		\item \texttt{name}
		\item \texttt{position}
		\item \texttt{salary}
	\end{itemize}

	\textbf{Запрос:}
	\begin{verbatim}
SELECT name, position
FROM employees;
\end{verbatim}

	\textbf{Описание:}
	Этот запрос вернёт столбцы \texttt{name} и \texttt{position} из таблицы \texttt{employees}.
\end{frame}

\begin{frame}[fragile]{Пример 2: Фильтрация (WHERE)}
	\begin{verbatim}
SELECT name, salary
FROM employees
WHERE salary > 50000;
\end{verbatim}
	\begin{itemize}
		\item Возвращает только тех сотрудников, у которых \texttt{salary} превышает 50000.
	\end{itemize}
\end{frame}

\begin{frame}[fragile]{Математические операции}
	\begin{itemize}
		\item Стандартные математические операции: \texttt{+, -, *, /, \%}
		\item Битовые операции: \texttt{\&, \^ \;, $\vert, \sim$}
		\item Операции сравнения: \texttt{>, <, <=, >=, =, <>, BETWEEN}
		\item Логические операции: \texttt{AND, OR, NOT, ALL, ANY/SOME, IN, LIKE, IS NULL, IS NOT NULL}
		\item Приведение типов: \texttt{cast(column\_name as type)}
	\end{itemize}
\end{frame}

\begin{frame}[fragile]{Пример 3: Использование математических операций}
	\begin{verbatim}
SELECT DISTINCT department
FROM employees
WHERE (name IN ('Саша', 'Маша', 'Петя')
       AND
       department = 'IT')
      OR salary / work_hours >= 5000;
    \end{verbatim}
	\texttt{DISTINCT} - уникальные значения в колонке
\end{frame}


\begin{frame}[fragile]{Пример 4: Сортировка (ORDER BY) и LIMIT}
	\begin{verbatim}
SELECT name, salary
FROM employees
ORDER BY salary DESC
LIMIT 3;
\end{verbatim}
	\begin{itemize}
		\item Сортирует сотрудников по убыванию \texttt{salary}.
		\item Показывает только первые 3 строки (самые высокие зарплаты).
	\end{itemize}
	\texttt{ASC} - сортировка по возрастанию, \texttt{DESC} - сортировка по убыванию
\end{frame}

\begin{frame}[fragile]{Aliasing (псевдонимы)}
	\begin{itemize}
		\item \textbf{Aliasing} позволяет давать временные имена столбцам или таблицам.
		\item Удобно при использовании вычисляемых полей или при работе с длинными названиями таблиц.
	\end{itemize}

	\textbf{Пример: Использование псевдонима столбца}
	\begin{verbatim}
SELECT name AS employee_name,
       salary * 1.2 AS new_salary
FROM employees;
\end{verbatim}
	\begin{itemize}
		\item \texttt{AS} задаёт псевдоним для отображаемого столбца.
		\item В результате в итоговой выборке столбец будет называться \texttt{employee\_name},
		      а второй --- \texttt{new\_salary}.
	\end{itemize}

	\textbf{Пример: Псевдоним таблицы}
	\begin{verbatim}
SELECT e.name, e.position
FROM employees AS e;
\end{verbatim}
\end{frame}

\begin{frame}[fragile]{Аггрегирующие функции}
	\begin{itemize}
		\item \texttt{MIN} - минимальное значение в колонке
		\item \texttt{MAX} - максимальное значение в колонке
		\item \texttt{AVG} - среднее значение в колонке
		\item \texttt{SUM} - сумма значений в колонке
		\item \texttt{COUNT} - количество значений в колонке
	\end{itemize}
	Важно: все аггрегирующие функции игнорируют \textbf{NULL}-значения кроме \texttt{COUNT}.
\end{frame}

\begin{frame}[fragile]{Пример 5: Группировка (GROUP BY) и HAVING}
	\begin{verbatim}
SELECT position, AVG(salary) AS avg_salary
FROM employees
GROUP BY position
HAVING AVG(salary) > 40000;
\end{verbatim}
	\begin{itemize}
		\item Группирует сотрудников по должности (\texttt{position}).
		\item Считает среднюю зарплату по каждой должности.
		\item Оставляет только те должности, где средняя зарплата больше 40000.
	\end{itemize}
\end{frame}

\begin{frame}[fragile]{Пример 6: Простые вложенные запросы}
	\begin{verbatim}
SELECT *
FROM employees
WHERE salary = (SELECT MAX(salary) FROM employees);
    \end{verbatim}
\end{frame}

\begin{frame}[fragile]{Средства работы с SQLite}
	\begin{itemize}
		\item SQLiteStudio
		\item PyCharm Professional Edition
		\item Visual Studio Code (SQL plugin)
		\item Vim/Neovim (dadbod plugin)
	\end{itemize}
\end{frame}

\section{Задания}

\begin{frame}[fragile]{Задания для factbook.db}
	\begin{itemize}
		\item Подключитесь к БД
		\item Сколько в ней таблиц?
		\item Для каждой таблицы в БД напишите запрос, получающий первые 10 строк любых двух колонок из этой таблицы
		\item Напишите запрос, получающий топ-20 столиц по населению
		\item Напишите запрос, получающий последние 10 строк любых двух колонок из таблицы facts
		\item Напишите запрос, получающий новую колонку \texttt{Population\_\%}, содержащую информацию о доле населения в городе относительно населения всех городов представленных в таблице
		\item Напишите запрос, получающий все строки, содержащие \textbf{NULL} хотя бы в одной колонке
		\item Выведите топ-10 стран по населению (facts\_id - идентификатор страны)
	\end{itemize}
\end{frame}

\begin{frame}[fragile]{Задания для jobs.db}
	\begin{itemize}
		\item Подключитесь к БД
		\item Напишите запрос, получающий все категории занятости (Major\_category), где сумма по всем сферам долей занятости женщин более 5
		\item Напишите запрос, получающий процентную долю мужчин в каждой сфере занятости (Major)
		\item Напишите запрос, получающий топ-10 сфер занятости по доли занятости женщин в этой сфере при условии что женщин не более 0.8 и не менее 0.5 от общего числа
		\item Напиште запрос, получающий все сферы занятости, где доля женщин более половины от общего числа либо менее, но при условии, что сфера занятости попадает в категорию \texttt{Engineering}
		\item Напишите запрос, получающий категории занятости, для которых показатель женской занятости (из задания 1) находится в промежутке от 3 до 5
	\end{itemize}
\end{frame}

\end{document}
